\documentclass[8pt]{beamer}% тип документа
\usepackage{caption}
\usepackage{cmap}
\usepackage[T2A]{fontenc}
\usepackage[utf8]{inputenc}
\usepackage[main=russian,english]{babel}
\usetheme{AnnArbor}
\usefonttheme{serif}
\geometry{paperwidth=140mm,paperheight=105mm}

\title{Разработка пособия "Структура данных «Дерево отрезков» и её применение в задачах"}
\author{Девятериков Иван}

% 1 направление проектной деятельности и тема проекта;
% 2 актуальность проекта
% 3 положительные эффекты от реализации проекта, важные как для самого автора, так и для других людей
% 4 ресурсы (как материальные, так и нематериальные), необходимые для реализации проекта, возможные источники ресурсов
% 5 риски реализации проекта и сложности, которые ожидают обучающегося при реализации данного проекта

\begin{document}% начало презентации
	
	\begin{frame}% первый слайд
		\titlepage
	\end{frame}
	
	\begin{frame}{Направление проекта}
		\begin{block}{Разработка пособия "Структура данных «Дерево отрезков» и её применение в задачах"}
			научно-технические исследование
		\end{block}
		
		
		\begin{block}{Задачи}
			\begin{itemize}
				\item Подобрать материал, в соответствии с рабочей программой.
				\item Определить структуру и содержание учебного пособия.
				\item Оформить теоретический материал согласно структуре.
				\item Составить задачи на данную тему и написать к ним разбор.
				\item Проверить составленное пособие на тестовой группе.
			\end{itemize}
		\end{block}
	
	\end{frame}

	\begin{frame}{Актуальность проекта}
		Методическое обеспечение является одной из важнейших составляющих учебного процесса, особенно в части самостоятельной работы школьников.
		
		Материал, представленный в сети интернет, часто ориентирован на студентов IT-специальностей. В то же самое время структура данных Дерево отрезков является универсальной и мощной структурой данных, для которой можно реализовать неограниченный набор операций. Знание такой структуры данных чрезвычайно полезно, в том числе при подготовке к олимпиадам высокого уровня.
		
	\end{frame}

	\begin{frame}{Эффекты от реализации проекта}
		\begin{block}{Для других людей}
			\begin{itemize}
				\item Изучение данной и смежной с ней тем.
				\item Умение решать сложные задачи по программированию.
			\end{itemize}
		\end{block}
	
		\begin{block}{Для меня лично}
			\begin{itemize}
				\item Погружение в тему проекта.
				\item Решение новых и интересных задач.
				\item Умении писать, думать, анализировать, планировать.
				\item Умение верстать книгу с помощью языка \LaTeX.
			\end{itemize}
		\end{block}
	\end{frame}

	\begin{frame}{Необходимое для реализации проекта}
		\begin{block}{Ресурсы}
			\begin{itemize}
				\item Компьютер.
			\end{itemize}
		\end{block}
	
		\begin{block}{Источники ресурсов}
			\begin{itemize}
				
				\item \href{https://codeforces.com}{codeforces} - площадка, где регулярно проводятся соревнования, система с более 6700 официальных задач
				
				\item \href{https://polygon.codeforces.com}{polygon} - сервис для подготовки задач по программированию.
				
				\item \href{https://codeforces.com/blog/entry/83170}{бакалаврская диссертация «Compressed segment trees and merging sets in $O(N \log U)$» } (Lucian Bicsi, Бухарестский университет)
				
				
				\item Книга «Конспект лекций по алгоритмам» (Сергей Копелиович, НИУ ВШЭ)
				
				\item \href{https://codeforces.com/edu/course/2}{Учебный курс} (Павел Маврин, Университет ИТМО)
				\item \href{https://codeforces.com/blog/entry/57319}{A simple introduction to "Segment tree beats"} (Ruyi Ji, Пекинский университет)
				
				\item \href{https://codeforces.com/blog/entry/18051}{Efficient and easy segment trees} (\href{http://finals.snarknews.info/index.cgi?data=2011/teams/knu&class=final2011&year=2011}{Александр Бачериков})
				
				\item \href{https://codeforces.com/blog/entry/15890}{Algorithm Gym :: Everything About Segment Trees} (AmirMohammad Dehghan, Массачусетский технологический институт)
				
				\item \href{https://codeforces.com/blog/entry/80195}{Matrix Exponentiation tutorial} (Kamil Debowski, Варшавский университет)
				
				\item \href{https://codeforces.com/blog/entry/43230}{Mo's Algorithm on Trees} (Animesh Fatehpuria, Технологический институт Джорджии)
				
				\item \href{https://codeforces.com/blog/entry/78898}{Tutorial on Permutation Tree} (Ashley Khoo)
				
				\item \href{https://codeforces.com/blog/entry/61364}{Searching Binary Indexed Tree in $O( \log(N))$ using Binary Lifting} (Siddharth Nayya, Делийский технологический университет)
				
				\item \href{https://e-maxx.ru/algo/segment_tree}{e-maxx.ru/algo/segment\_tree}
				
				\item \href{https://codeforces.com/blog/entry/15729}{много структур}
				
				\item Так же есть Китайские форумы (я нашёл китайский аналог блогов на кф'е и там один миллиард китайцев решают задачи с cf) (\href{https://blog.csdn.net/weixin_43826249/article/details/102600666}{Вот пример просто транслейт или учите Китайский})
				
			\end{itemize}
		\end{block}
		
		
	\end{frame}





	\begin{frame}{Реализация данного проекта}
		\begin{block}{Риски}
			\begin{itemize}
				\item -
			\end{itemize}
		\end{block}
		
		\begin{block}{Cложности}
			\begin{itemize}
				\item Изучение вёрстки в \LaTeX.
				\item Анализ книг и статей на русском, английском и др.
				\item Написание или поиск подходящих задач.
			\end{itemize}
		\end{block}
	\end{frame}






\end{document}